\section{Tresortering}\label{treesort}
I likhet med \nameref{heapsort} utnytter vi her egenskapene til en datastruktur,
nemlig binære søketrær. I et slikt tre vet vi at venstre barn til en node er
mindre enn noden selv, og at høyre barn er større eller lik noden.
Deretter traverserer vi treet i infix-rekkefølge. På den måten vil vi altså gå
gjennom treet i stigende rekkefølge, og vi har dermed sortert arrayen.

\paragraph{Kompleksitet}
Fra teorem \ref{teo:bintre} har vi at innsetting i et binært søketre tar $ O(\log_2 n) $ beste tid, og $ O(n) $ verste tid. En traversering av treet tar opplagt $ O(n) $ tid, siden vi går gjennom hver node én gang. Tilsammen har vi da at kjøretiden for algoritmen er 
\begin{align*}
\text{I beste fall / gjennomsnittlig:\quad} & T(n) = O(n\log_2 n)\\
\text{I verste fall:\quad} & T(n) = O(n\cdot n) = O(n^2)
\end{align*}