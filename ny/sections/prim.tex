\section{Prims algoritme}
\label{prim}

Prims algoritme er en algoritme for å finne minimale spenntrær. Det er en grådig algoritme siden den bygger treet trinnvis ved å velge den kanten som går ut av treet med lavest vekt. 

\begin{theorem} Prims algoritme

Vi skal finne et minimalt spenntre for en graf $ G $. Velg først en startnode (rotnode). Den kan velges helt tilfeldig. Så lenge treet ikke spenner hele $ G $ legger vi til den kanten fra treet, til en ukjent node som har lavest vekt.
\end{theorem}
\noindent Vi ser på et eksempel.
\begin{example} Finn et minimalt spenntre for grafen i figur \ref{fig:min_spenntre}.

Vi velger $ v_1 $ som rotnode (vi kunne valgt hvilken som helst annen). Den kanten med lavest vekt ut fra treet (som kun består av $ v_1 $) er kanten $ v_1-v_4 $, så vi legger den til. Videre er kanten $ v_1-v_2 $ den kanten med lavest vekt som går til en ukjent node, så vi legger til den. $ v_4-v_3 $ er neste kant vi legger til. Kanten $ v_2-v_4 $ er nå den kanten med lavest vekt som går ut fra treet, men siden $ v_2 $ og $ v_4 $ begge er inneholdt i treet er ikke den interessant. Neste kant vi legger til er $ v_4-v_7 $, deretter $ v_7-v_6 $ og $ v_7-v_5 $. Treet er nå et komplett spenntre for $ G $. 
\end{example}

\subsection{Tidsanalyse}
Samme argument som tidsanalysen til Dijkstra. Total kjøretid: $ O\left(|V|^2\right) $
