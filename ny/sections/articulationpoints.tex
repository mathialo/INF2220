\section{Articulation points og biconnectivity}
Et \textbf{articulation point} er et kritisk punkt i en sammenhengende graf som ikke kan fjernes uten at grafen ikke lenger er sammenhengende. I grafen i figur \ref{fig:graf} er C et slikt punkt. Vi ser at vi kan ta vekk A uten noen store komplikasjoner, men hvis vi tar vekk C vil ikke lenger B og E være sammenknyttet.\index{graf!articulation points}

En \textbf{biconnected} graf er en graf uten articulation points, altså en graf der det alltid finnes mer enn én vei fra en node til en annen. Grafen (til venstre) i figur \ref{fig:min_spenntre} er en slik graf. \index{graf!biconnected}