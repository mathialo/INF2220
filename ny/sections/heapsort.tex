\section{Heapsort}\label{heapsort}
I heapsort utnytter vi egenskapene til en heap.
Her bruker vi en max-heap, som vi setter alle elementene inn i. Da vet vi at
rota er det største elementet per detfinisjon av heap. Dersom vi så tar ut et og
et element av heapen, og plasserer på den bakerste ikke-opptatte plassen i
resultat-arrayen, vil vi få arrayen ut i sortert form.

\paragraph{Kompleksitet}
Fra teorem \ref{teo:heapop} har vi at innsetting og sletting i en heap tar, i verste fall, $ O(\log_2 n) $ tid. I beste fall har vi at innsetting tar $ O(1) $ tid, og at sletting tar $ O(\log_2 n) $ tid. Siden lista er $ n $ elementer lang må vi gjøre $ n $ innsettinger og $ n $ slettinger. Det gir at total kjøretid er
\begin{align*}
\text{I beste fall:\quad} & T(n) = n + n\log_2 n = O(n\log_2 n)\\
\text{I verste fall:\quad} & T(n) = n\log_2 n + n\log_2 n = O(n\log_2 n)
\end{align*}