\section{Floyds algoritme}
\label{floyd}

Floyds algoritme er en algoritme som beregner korteste vei fra alle noder til alle andre noder. Algoritmen er et eksempel på en algoritme som benytter seg av dynamisk programmering. 

\begin{theorem} Floyds algoritme

Denne betraktningen gjentas systematisk for alle tripler i, k og j:
\begin{itemize}
\item Initielt: Avstanden fra node $ i $ til node $ j $ settes lik vekten på kanten fra $ i $ til $ j $, uendelig hvis det ikke går noen kant fra $ i $ til $ j $
\item Trinn 0: Se etter forbedringer ved å velge node 0 som mellomnode
\item Etter trinn $ k $: Avstanden mellom to noder er den korteste veien som bare bruker nodene $ 0, 1, ... , k $ som mellomnoder
\end{itemize}
\end{theorem}

\noindent Eksempel på implementasjon i Java:\newpage
\javaimport{code/floyd.java}

\subsection{Tidsanalyse}
Vi ser fra kodeeksempelet at det er to løkker som må kjøres gjennom. En dobbel og en trippel for-løkke. Kjøretiden blir derfor:

\[ \text{Kjøretid} ~=~ n^2 + n^3  ~=~ O\left(n^3\right) \]

\noindent Floyds algoritme har forøvrig samme kompleksitet som å kjøre \nameref{dijkstra} én gang for alle nodene i grafen. Fordelen med Floyds algoritme er at den er litt mer robust i møte med løkker og negative vekter. 


\subsection{Hvordan tolke input/output fra Floyds algoritme}
Ved start inneholder \mono{nabo[i][j]} lengden av kanten fra $ i $ til $ j $, \mono{nabo[i][j]}= |$ \infty $ hvis det ikke er noen kant. Algoritmen lar \mono{nabo} være uendret, og legger resultatene i \mono{avstand} og \mono{vei}. \mono{avstand[i][j]} er lengden av korteste vei fra $ i $ til $ j $. Når vi oppdager at den korteste veien fra $ i $ til $ j $ går gjennom $ k $ setter vi \mono{vei[i][j] = k}. \mono{vei} sier derfor om den korteste veien. Vi finner korteste vei slik:
\begin{itemize}
\item \mono{k1 = vei[i][j]} er den største verdien av $ k $ slik at $ k $ ligger på den korteste veien fra $ i $ til $ j $
\item \mono{k2 = vei[i][k1]} er den største verdien av $ k $ slik at $ k $ ligger på den korteste veien fra $ i $ til $ k_1 $
\\ $ \vdots $
\item Når \mono{vei[i][km] = -1} er $ (i, k_m) $ den første kanten i korteste vei fra $ i $ til $ j $
\end{itemize}
Dermed er korteste vei fra $ i $ til $ j $ slik: $ i \rightarrow k_m \rightarrow k_{m-1} \rightarrow ... \rightarrow k_2 \rightarrow k_1 \rightarrow j $. 