\label{sort_form}
Før vi kan begynne å se på noen spesielle sorteringsalgoritmer må vi formalisere hva vi mener med sortering. Vi definerer sortering slik:

\begin{definition}
Anta at \mono{a[]} er en liste av sammenlignbare elementer. Vi sier at \mono{a'[]} er den tilhørende sorterte lista hvis følgende kriterier er oppfylt:
\begin{enumerate}[i]
\item \mono{a'[i]} $ < $ \mono{a'[i+1]} for alle  \mono{i}$ = 0, 1, \ldots, n-1 $
\item Alle elementene i \mono{a[]} er med i \mono{a'[]}
\end{enumerate}
\end{definition}

Det andre kriteriet kan virke litt snodig, men uten det ville sortering vært veldig enkelt. Vi kunne i så fall bare generert en ny liste med elementer i sortert rekkefølge, og det første kriteriet ville vært oppfylt. Vi trenger derfor bevaringskriteriet. 

Med ``sammenlignbare'' mener vi at det finnes en måte å entydig bestemme om et element er større enn, mindre enn eller lik et annet element. Hvis vi skal sortere tall er jobben enkel: vi sammenligner numerisk verdi. Hvis vi skal sammenligne to tekststrenger er det ikke like opplagt hvordan vi skal gjøre det. Skal vi sortere alfabetisk? Etter lengde på ordet? I et sånt tilfelle er det opp til oss å velge et fornuftig sammenligningskriterie. Det er vilkårlig hvordan vi sammenligner to elementer, så lenge vi gjør det likt gjennom hele sorteringen. 

\subsection*{Paradigmer og begreper}
\label{sec:sortbegrep}

\paragraph{Klasser av sorteringsalgoritmer}
Avhengig av hvordan de virker, deler vi sorteringsalgoritmer inn i to klasser:
Sammenligningsbaserte og Verdibaserte. Førstnevnte baserer seg på å sammenligne
to og to verdier i arrayen, mens de verdibaserte finner den riktige plasseringen
til hvert element kun ut ifra verdien på elementet. 


\paragraph{Stabil Sortering}\label{stabil}
Noen ganger ønsker vi å sortere etter flere kriterier. Et eksempel kan være
epost, der vi f.eks. vil sortere både etter avsender og alfabetisk, slik at vi
får at alle som er fra samme avsender står alfabetisk i forhold til hverandre.
\newpage
\begin{definition}
  En algoritme er \textbf{stabil} dersom innbyrdes rekkefølge beholdes
  etter sortering. Dvs at dersom \mono{a[i]} $ = $ \mono{a[j]}, og \mono{a[i]} kommer før \mono{a[j]} i \mono{a[]}, så skal \mono{a[i]} komme før \mono{a[j]} i den sorterte lista \mono{a'[]}.
\end{definition}

Følgende algoritmer (altså ikke alle) gjennomgått i kurset oppfyller kravet om
stabil sortering:
\begin{itemize}
\item \nameref{insertsort} 
\item \nameref{radixsort} 
\item \nameref{mergesort} kan være stabil, avhengig av hvordan flettemetoden er implementert. I vårt (og bokas) eksempel er det en stabil sortering.
\end{itemize}


\paragraph{Nedre grense for enkle sorteringsalgoritmer}
\begin{theorem} (7.1 i læreboka): Det gjennomsnittlige antall inversjoner i en \label{teo:sortbound}
  array med $n$ unike elementer er $\frac{n(n-1)}{4}$
\end{theorem}
Beviset bruker at summen av en aritmetisk rekke er $ n(n-1)/2 $:
\begin{proof}
  La $L$ være en array og $L_r$ den reverserte. Et hvilket som helst ordnet par
  i $L$ vil være en inversjon i enten $L$ eller $L_r$. Det totale antallet slike
  par i $L$ og $L_r$ vil være $\frac{n(n-1)}{2}$, altså har en gjennomsnittlig
  liste halvparten så mange, altså $\frac{n(n-1)}{4}$ 
\end{proof}
Vi har da et korollar til teorem \ref{teo:sortbound} som gir oss gjennomsnittlig tid for algoritmer som bytter naboelementer:
\begin{corollary} (7.2 i læreboka): \label{teo:swapkompl}
  Enhver sorteringsalgoritme som sorterer ved å bytte to naboelementer krever i gjennomsnitt $ O(n^2) $ tid.
\end{corollary}
\begin{proof}
Fra teorem \ref{teo:sortbound} har vi at gjennomsnittlig antall inversjoner er $ n(n-1)/4 $. Hvert plassbytte av to naboer inversjon fjerner én inversjon, dermed tar algoritmer som baserer seg på plassbytte av naboer gjennomsnittlig $ O(n^2) $ tid.
\end{proof}