\section{Rotasjon}
\label{trerotasjon}

\subsection{Zig}
For å skjønne zig-rotasjon ser vi på en figur:

\begin{figure}[H]
\caption{Zig-rotasjon mhp X}
\centering
\begin{tikzpicture}[level distance=1.5cm,
  level 1/.style={sibling distance=2cm},
  level 2/.style={sibling distance=1.3cm}]

\tikzstyle{subtree} = [isosceles triangle, draw=black, align=center, minimum height=0.5cm, minimum width=1cm, shape border rotate=90, anchor=center]
\tikzstyle{n} = [draw=black, draw=black, align=center, circle]

\node[n] {P}
child {
	node[n] {X}
	child {
		node[subtree] {A}
	}
	child {
		node[subtree] {B}
	}
}
child {
	node[subtree] {C}
}
;
\end{tikzpicture}
\quad\raisebox{5\height}{\scalebox{2}{$\rightarrow$}}\quad
\begin{tikzpicture}[level distance=1.5cm,
  level 1/.style={sibling distance=2cm},
  level 2/.style={sibling distance=1.3cm}]

\tikzstyle{subtree} = [isosceles triangle, draw=black, align=center, minimum height=0.5cm, minimum width=1cm, shape border rotate=90, anchor=center]
\tikzstyle{n} = [draw=black, draw=black, align=center, circle]

\node[n] {X}
child {
	node[subtree] {A}
}
child {
	node[n] {P}
	child {
		node[subtree] {B}
	}
	child {
		node[subtree] {C}
	}
}
;
\end{tikzpicture}
\end{figure}

En zig-rotasjon vil beholde egenskapene til et binært søketre siden alle nodene i A er mindre enn X og P, alle nodene i B er større enn X og mindre enn P, og alle nodene i C er større enn både X og P. Vi ser at alle høyre barn av P også er høyre barn av P \emph{etter} rotasjonen, etc. 

Vi har også noe som heter zig-zig-rotasjon, som er å utføre en zig-rotasjon to ganger.


\subsection{Zig-zag}
Igjen ser vi på en figur:

\begin{figure}[H]

% Dette er helt forferdelig kode. Jeg vet, og jeg beklager. 

\caption{Zig-zag-rotasjon mhp X:}
\centering
\begin{tikzpicture}[level distance=1.5cm,
  level 1/.style={sibling distance=2cm},
  level 2/.style={sibling distance=1.3cm}]

\tikzstyle{subtree} = [isosceles triangle, draw=black, align=center, minimum height=0.5cm, minimum width=1cm, shape border rotate=90, anchor=center]
\tikzstyle{n} = [draw=black, draw=black, align=center, circle]

\node[n] {G}
child {
	node[n] {P}
	child {
		node[subtree] {A}
	}
	child {
		node[n] {X}
		child {
			node[subtree] {B}
		}
		child {
			node[subtree] {C}
		}
	}
}
child {
	node[subtree] {D}
}
;
\end{tikzpicture}
\quad\raisebox{9\height}{\scalebox{2}{$\rightarrow$}}\quad
\raisebox{.25\height}{
\begin{tikzpicture}[level distance=1.5cm,
  level 1/.style={sibling distance=2.7cm},
  level 2/.style={sibling distance=1.3cm}]

\tikzstyle{subtree} = [isosceles triangle, draw=black, align=center, minimum height=0.5cm, minimum width=1cm, shape border rotate=90, anchor=center]
\tikzstyle{n} = [draw=black, draw=black, align=center, circle]

\node[n] {X}
child {
	node[n] {P}
	child {
		node[subtree] {A}
	}
	child {
		node[subtree] {B}
	}
}
child {
	node[n] {G}
	child {
		node[subtree] {C}
	}
	child {
		node[subtree] {D}
	}
}
;
\end{tikzpicture}
}
\end{figure}

Som i tilfellet med zig-rotasjon ser vi at egenskapene til et binært søketre er bevart. 