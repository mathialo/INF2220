\section{Boyer-Moore(-Horspool)}
Dette er en rask substringalgoritme. Med Boyer-Moore preprosserer vi nåla før vi begynner å søke. Vi regner ut hvor mange hopp vi kan gjøre for hvert enkelt tegn vi får mismatch på. Dette kaller vi \textit{bad character shift}. Vi beregner også \textit{good character shift}, som er verdier vi kan hoppe på et gitt sted i nåla hvis vi har hatt match tidligere i nåla, men ikke på denne plassen. 
\newpage
\begin{theorem} Boyer-Moore
	\begin{enumerate}
		\item Beregn bad character shift for alle tegn.
		\item Sammenlign nåla med teksten, starter med tegnet lengst til høyre i nåla.
		\item Hvis mismatch på \mono{c}, flytt nåla fram med den \textbf{største} verdien av \mono{badCharShift[c]} og \mono{goodSuffixShift[c]}. Hvis match, dytt nåla en til høyre og sjekk neste tegn. Gjenta.
	\end{enumerate}
\end{theorem}
Merk at når vi snakker om å skippe så er det alltid i forhold til det siste tegnet i nåla. Vær også obs på, som det står i algoritmen, at når vi bruker denne vil vi skippe maks av \mono{badCharShift[c]} og \mono{goodSuffixShift[c]}, der \mono{c} er tegnet vi sammenligner.

\subsection{Bad character shift}
Vi beregner avstand til neste gang nål er på linje med høystakk.
Ved en mismatch vil vi dytte nåla framover til en av to ting skjer: Mismatch blir match eller nåla beveger seg forbi tegnet vi fikk mismatch på.
I praksis gjør vi dette slik: Verdien til et tegns bad character shift er lengden på nåla minus den siste indeksen som tegnet befinner seg på, minus 1. Altså \mono{value = length - index - 1}.

\begin{example}
	La nåla være ``tooth'', og teksten være ``trusthardtoothbrushes''.
	
	\begin{center}
		\begin{tabular}{r | c c c c c}
			& T & O & O & T & H \\ \hline
			index & 0 & 1 & 2 & 3 & 4
		\end{tabular}
	\end{center}
	
	\noindent Vi konstruerer bad character shift for denne med regelen \mono{value = length - index - 1}.
	
	\begin{center}
		\begin{tabular}{cccl}
			T = & $5-0-1$&= 4\\
			O = & $5-1-1$&= 3\\
			O = & $5-1-2$&= 2&Erstatter her forrige verdi av O med ny verdi til O.\\
			T = &$5-3-1$& = 1&Erstatter her forrige verdi av T med ny verdi til T.\\
			H = &5 &&Verdien skal ikke være mindre enn 1. Får da verdi lik lengden.\\
		\end{tabular}
	\end{center}
	\noindent Vi sitter igjen med dette som bad character shift-tabell:
	\begin{center}
		\begin{tabular}{r|cccc}
			Bokstav&T&O&H&$*$\\
			\hline
			Verdi&1&2&5&5\\
		\end{tabular}
	\end{center}
	
	\noindent Vi bruker nå bad character shift på eksemplet vårt: \vspace{10pt}\\
	\scalebox{0.9}{
		\begin{tabular}{c|ccccccccccccccccccccc}
			&T&R&U&S&\textcolor{red}{T}&\textcolor{red}{H}&A&R&D&T&\textcolor{red}{O}&O&\textcolor{red}{T}&H&B&R&U&S&H&E&S\\\hline
			\textbf{1.}&T&O&O&T&\textcolor{red}{H}\\
			\textbf{2.}&&T&O&\textcolor{red}{O}&\textcolor{dkgreen}{T}&\textcolor{dkgreen}{H}\\
			\textbf{3.}&&&&&&&T&O&O&T&\textcolor{red}{H}\\
			\textbf{4.}&&&&&&&&&T&O&O&T&\textcolor{red}{H}\\
			\textbf{5.}&&&&&&&&&&\textcolor{dkgreen}{T}&\textcolor{dkgreen}{O}&\textcolor{dkgreen}{O}&\textcolor{dkgreen}{T}&\textcolor{dkgreen}{H}\\
		\end{tabular}
	}
	\vspace{10pt}\\
	I steg \textbf{1.} får vi mismatch på H, fordi den ikke er det samme som T. Da må vi slå opp i tabellen på den bokstaven som \textbf{\textit{vi møter i teksten}}, og hoppe fram tilsvarende antall steg. I første tilfellet skal vi hoppe 1 plass fram (for 1 er verdien til T). 
	
	Sjekker på nytt i steg \textbf{2.}, her får vi match på T og H, men mismatch på O. Da hopper vi H-plasser frem.
	
	I steg \textbf{3.} får vi mismatch på H mot O, og må hoppe O-plasser frem---altså 2. I steg \textbf{4.} er det mismatch mellom H og T, vi hopper T-plasser frem---altså 1. Og nå har vi funnet nåla vår!
	
\end{example}

%\paragraph{Analyse}
%Worst case er det samme som brute force. Input tekst $1^n$ kjører $n$ ganger, og nål $011\dots1$ kjører $m$ ganger. Dette gir O($nm$). Best case har input tekst $1^n$ og nål $0^m$, som gir O($n/m$). Average case er O($m/|\Sigma|$), altså raskere enn brute force.

\subsection{Good suffix shift}
Anta at vi har en substring $t$ av nåla (altså lengst til høyre i nåla) som allerede er matchet. Vi får så mismatch på neste tegn. Vi kan nå være smarte og finne ut, og
vi må tenke på to tilfeller: $t$ forekommer et annet sted i nåla. Da kan vi ikke skippe lenger enn til neste gang det skjer. Eller, en del av $t$ forekommer i starten av nåla. Da må vi skippe til vi er på linje med dette.

\begin{example}
	Vi ser på nåla ``TTCTATTCTT''.
	\begin{enumerate}
		
		\item Sjekker først \textit{ikke}-\mono{T}, det er to shift før vi finner dette.
		\begin{align*}
			\text{\mono{TCCTATTCT}}&\text{\mono{{\textcolor{red}{T}}}}\\
			\text{\mono{TCCTATT}}&\text{\mono{\textcolor{red}{C}TT}}
		\end{align*}
		
		\item Så sjekker vi \textit{ikke}-\mono{TT}. Dette finner vi etter ett shift.
		\begin{align*}
			\text{\mono{TCCTATTC}}&\text{\mono{\textcolor{red}{TT}}} \\
			\text{\mono{TCCTATT}}&\text{\mono{\textcolor{red}{CT}T}}
		\end{align*}
		
		\item For å finne \textit{ikke}-\mono{CTT} må vi flytte 3 steg; til \mono{ATT}.
		\begin{align*}
		\text{\mono{TCCTATT}}&\text{\mono{\textcolor{red}{CTT}}} \\
		\text{\mono{TCCT}}&\text{\mono{\textcolor{red}{ATT}CTT}}
		\end{align*}
		
		
		\item Så kommer vi til \textit{ikke}-\mono{TCTT}. Denne eksisterer ikke, MEN om nålen har lik suffix som prefiks (her \mono{T} som start og slutt), så flytter vi bare fram til prefiksen. Da må vi flytte nålen 9 hakk, selv om lengden er 10. 
		\begin{align*}
		\text{\mono{TCCTATTCT}}&\text{\mono{\textcolor{red}{T}}} \\
		&\text{\mono{\textcolor{red}{T}}}\text{\mono{CCTATTCTT}}
		\end{align*}
		
		
	\end{enumerate}
	
	~\\Alle de neste shiftene vil også være dette, fordi de får ingen andre treff i nålen. Vi får da følgende good suffix table:
	\begin{figure}[H]
		\centering
		\begin{tabular}{crc}
			index&mismatch&shift\\
			\hline
			0&\mono{\textcolor{red}{T}}&2\\
			1&\mono{\textcolor{red}{T}T}&1\\
			2&\mono{\textcolor{red}{C}TT}&3\\
			3&\mono{\textcolor{red}{T}CTT}&9\\
			4&\mono{\textcolor{red}{T}TCTT}&9\\
			5&\mono{\textcolor{red}{A}TTCTT}&9\\
			6&\mono{\textcolor{red}{T}ATTCTT}&9\\
			7&\mono{\textcolor{red}{C}TATTCTT}&9\\
			8&\mono{\textcolor{red}{C}CTATTCTT}&9\\
			9&\mono{\textcolor{red}{T}CCTATTCTT}&9\\
		\end{tabular}
	\end{figure}
\end{example}
