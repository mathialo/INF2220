\section{Radixsort}\label{radixsort}
Radix sort sorterer en array på et siffer av gangen.
Metoden forekommer i to varianter, en fra høyre til venstre, og en som gjør
motsatt. Generelt for begge algoritmene har vi
\begin{enumerate}
\item Finn max verdi i arrayen. (Største siffer i alle tallene)
\item\label{item:tell} Tell opp hvor mange elementer det er av hver verdi av det sifferet vi ser
  på. (hvor mange 0-er, 1-ere osv)
\item Ved å addere disse antallene, finner vi ut hvor i arrayen tallene skal stå
  ved å addere verdiene i arrayen vi lagde i~\ref{item:tell}, men starte på 0.
\end{enumerate}
En spesiell ting ved Radix sort er at den aldri gjør sammenligninger mellom to
tall.

\paragraph{Right-Radix (RR)}
Den ``vanlige'' formen for Radix-sort, og er iterativ.
Den begynner med det minst gjeldende sifferet (det bakerste). Etter
``fellesstegene'' blir den siste arrayen vi lagde brukt til å slå opp hvor vi
skal plassere elementet. Vi går gjennom arrayen fra venstre til høyre, og setter
inn elementer i den nye arrayen. At vi gjør det fra høyre til venstre er viktig,
siden vi ønsker en~\hyperref[stabil]{stabil} sortering. Etter at vi har plassert
et element på sin plass i den nye ``sorterte'' arrayen, bør vi øke verdien vi
akkurat brukte i telle-arrayen med 1 så vi lett kan finne ut hvor neste tall med
samme siffer på gjeldende plass. Dette gjentar vi helt til vi har vært gjennom
maks antall gjeldende siffer.

\begin{example}
  \begin{displaymath}
    \begin{array}{c c c c c c c c}
      170&45&75&90&802&2&24&66
    \end{array}
  \end{displaymath}
  Teller deretter opp antall tall som slutter på 0, 1 osv. resulterer i
  \begin{displaymath}
    \begin{array}{c c c c c c c c c c}
      2&0&2&0&1&2&1&0&0&0
    \end{array}
  \end{displaymath}
  Altså har vi to tall som slutter på 0, to som slutter på 2 osv. For å finne
  den nye plasseringen til elementene, summerer vi opp plassene i denne arrayen,
  og får
  \begin{displaymath}
    \begin{array}{c c c c c c c c c c}
      0&2&2&4&4&5&7&8&8&8
    \end{array}
  \end{displaymath}
  Da vet vi f.eks. at tall som slutter på 0 skal settes inn på indeks 0. Siden
  vi har to slike, kan vi ikke sette inn de som slutter på 2 noe lenger ut i
  arrayen enn indeks 2. Når vi setter inn, og beholder innbyrdes ordning, får
  vi.
  \begin{displaymath}
    \begin{array}{c c c c c c c c}
      170&90&802&2&24&45&75&66
    \end{array}
  \end{displaymath}
  Deretter gjentar vi prosesen men denne arrayen, og sorterter på siffer nr 2
  fra høyre. Dette gjentar vi til vi har vært gjennom så mange gjeldende siffer
  som det største tallet i arrayen har. I dette tilfellet 3 (siden 802 har tre
  gjeldende siffer.) Merk at f.eks. 2 er å betrakte som 002 videre i sorteringen.
\end{example}

\paragraph{Kompleksitet}
I radix sort gjør vi kun lineære operasjoner: Først går vi gjennom lista for å
finne den største verdien, deretter går vi gjennom lista for å telle, så går vi
gjennom for å plassere elementene på sine nye plasser. De siste to punktene
gjøres like mange ganger som vi har gjeldende sifre. La oss kalle denne verdien
for $k$. Da får vi $O(n + 2kn)$, som er lik $O(kn)$ siden $k\geq 1$. Alternativt
kan vi betrakte $k$ som en konstant, og vi får at radix sort er $O(n)$