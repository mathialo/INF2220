\section{Noen bevisteknikker}

\paragraph{Bevis ved selvmotsigelse}
Teknikken går ut på å gjøre noen antagelser, og vise til at det fører til en motsigelse. Da må noen av antagelsene være gale, og hvis vi kun har to muligheter (for eksempel rett/galt) kan vi konkludere med at det andre alternativet må være rett. Eksempel på et slik bevis er beviset for haltingproblemet (\ref{haltingProof})

\paragraph{Induksjon}
Bevis ved induksjon går ut på å anta at påstanden holder for alle mulige tall $ k $ opp til $ n $, og vise at det også medfører at det også holder for $ k = n+1 $. Dermed har man vist at hvis det også holder for $ n+1 $ vil det holde for $ n+2 $, også videre. Det siste man må gjøre er å \say{starte} induksjonen, med for eksempel å vise at påstanden holder for $ k=1 $ (da vil den også holde for $ k=2, 3, ... $)

\paragraph{Bevis ved moteksempel}
Skal man vise at en påstand er falsk, kan man ganske enkelt finne et eksempel som viser at den opprinnelige påstanden er falsk. Påstår jeg at alle hester er hvite, kan du motbevise den påstanden ved å vise meg en hest som er svart.
