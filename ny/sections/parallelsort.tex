\section{Parallell Sortering} \vspace{-20pt}
\subsection{Amdahls lov}
\label{sec:amdahl}

Amdahls lov gir oss en slags øvre grense for hvor mye tid vi kan spare på å
parallellisere en algoritme.
\begin{theorem}\textbf{Amdahls lov: } 
  Anta at du har en algoritme der en andel
  $p\%$ må kjøres sekvensielt, men at resten kan gjøres i parallell.
  Med $k$ lik antall prosessorer, får vi at maksimal speedup er
  \[
    S = \frac{\text{Sekvensiell tid}}{\text{Parallell tid}} \leq \frac{100}{p + \frac{100 - p}{k}}
  \]
\end{theorem}

Denne loven har imidlertid møtt motstand i Gustafson, som hevder at andelen av
programmet som er sekvensielt blir mindre etterhvert som problemet blir større.
Altså er ikke Amdahls lov fast.

\subsection{Parallell Quicksort}\label{parquick}
Siden \nameref{quick} deler lista opp i to deler der alle elementene i den ene lista er mindre enn alle elementene i den andre lista er den velegnet til å parallellisere. Vi må gjøre de første rekursjonene sekvensielt og deler opp lista i $ n $ biter (merk at $ n $ på være på formen $ 2^k, k\in \mathbb{N} $). Deretter starter vi $ n $ tråder med hver sin del å sortere. For å $ n $ tråder må vi gjøre $ \log_2 n $ nivåer med rekursjon sekvensielt først. Når alle trådene er ferdig har vi $ n $ sortere biter, der vi vet at alle elementene i tråd 1 er mindre enn elementene i tråd 2, som er mindre enn elementene i tråd 3, osv. Da kan vi sette sammen bitene sekvensielt til slutt og få en sortert liste. 
\begin{figure}[H]
\caption{Illustrasjon av parallell quicksort med to tråder}
\centering
Utgangspunkt:\\
\begin{tabular}{| c |c| c |}
\hline
\quad\quad\quad\quad...\quad\quad\quad\quad&$ p $&\quad\quad\quad\quad...\quad\quad\quad\quad\\
\hline
\end{tabular}\\
~\\$ p $ som pivot, deler opp:\\
\begin{tabular}{| c |}
\hline
\quad\quad\quad\quad$ < p $\quad\quad\quad\quad\quad\\
\hline
\end{tabular}\quad\quad
\begin{tabular}{| c |}
\hline
\quad\quad\quad\quad$ \geq p $\quad\quad\quad\quad\quad\\
\hline
\end{tabular}\\
~\\ Starter egne tråder som sorterer disse to bitene, setter sammen til slutt.
\end{figure}

