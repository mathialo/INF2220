\section{Bevisføring}
\subsection{Noen bevisteknikker}
\paragraph{Bevis ved selvmotsigelse}
Teknikken går ut på å gjøre noen antagelser, og vise til at det fører til en motsigelse. Da må noen av antagelsene være gale, og hvis vi kun har to muligheter (for eksempel rett/galt) kan vi konkludere med at det andre alternativet må være rett. Eksempel på et slik bevis er beviset for haltingproblemet (\ref{haltingProof})

\paragraph{Induksjon}
Bevis ved induksjon går ut på å anta at påstanden holder for alle mulige tall $ k $ opp til $ n $, og vise at det også medfører at det også holder for $ k = n+1 $. Dermed har man vist at hvis det også holder for $ n+1 $ vil det holde for $ n+2 $, også videre. Det siste man må gjøre er å \say{starte} induksjonen, med for eksempel å vise at påstanden holder for $ k=1 $ (da vil den også holde for $ k=2, 3, ... $)

\paragraph{Bevis ved moteksempel}
Skal man vise at en påstand er falsk, kan man ganske enkelt finne et eksempel som viser at den opprinnelige påstanden er falsk. Påstår jeg at alle hester er hvite, kan du motbevise den påstanden ved å vise meg en hest som er svart.



\subsection{Reduksjon}
Reduksjon er å ta et ukjent problem, og \say{omforme} det til noe kjent. Vi ser på et eksempel:

Anta at vi har et problem $ A $, og vi lurer på om problemet er løselig eller ikke. $ A $ kan være latterlig komplekst og vanskelig å forstå, men kanskje vi kan uttrykke det på en annen måte? Hvis vi kan vise at hvis $ A $ er løselig er også $ B $ løselig, og hvis $ B $ er løselig er også $ C $ løselig. Slik kan vi fortsette til vi kommer til noe kjent, for eksempel haltingproblemet (\ref{haltingProof}). For matematikere:
\[ \text{$ A $ er løselig} \Rightarrow \text{$ B $ er løselig} \Rightarrow ... \Rightarrow \text{haltingproblemet er løselig} \]
Hvis vi kan vise en slik rekke av implikasjonspiler har vi \emph{redusert} $ A $ til haltingproblemet, og dermed vet vi at $ A $ er uløselig.

\subsection{Noen beviser}
\subsubsection{Haltingproblemet}
Kan en Turingmaskin avgjøre om en annen Turingmaskin noen sinne vil stoppe, eller om den vil gå i evig loop, om den ser på inputen til maskinen? Som nevnt i \ref{språk} kan vi definere en språk som mengden av input som vil få en maskin til å stoppe:\index{haltingproblemet}
\[ L = \{i : M \text{ stopper på input } i \} \]
Det kan vises at en Turingmaskin som løser dette problemet \emph{ikke} kan eksistere. Det kan vises på flere måter, Alan Turing beviste det slik:

\label{haltingProof}
Vi skal bevise at haltingproblemet er uløselig, og vi skal gjøre det ved selvmotsigelse. 

Anta at vi har en Turingmaskin $ M $ som bestemmer om en annen Turingmaskin $ H $ vil stoppe, gitt input $ i $. Vi kan da bygge en annen Turingmaskin $ M' $ rundt denne som tar de samme parametrene ($ H $ og $ i $) som input, og gir resultat hvis $ M $ gir false (Altså at $ H $ ikke vil stoppe, gitt input $ i $), og går i en evig loop hvis $ M $ gir true (at $ H $ vil stoppe, gitt $ i $). 

Hva vil skje hvis vi bruker denne Turingmaskinen på seg selv? Hvis $ M $ sier at $ M' $ vil stoppe vil $ M' $ gå i en evig loop, og følgelig vil ikke $ M' $ stoppe. Hvis $ M $ sier at $ M' $ ikke vil stoppe vil $ M' $ gi et resultat, og så stoppe. Uansett får vi en motsigelse, og derfor kan ikke Turingmaskinen $ M $ eksistere. $ \hfill\qed $ \\

\noindent Beviset er analogt med Cantors diagonaliseringsargument.

\subsubsection{Cantors diagonaliseringsargument}
Vi skal vise at størrelsen av $ \mathbb{R} $ er større enn $ \mathbb{N} $, altså at det er flere reelle tall enn naturlige tall. Dette er et eksempel på bevis ved selvmotsigelse. Vi skal anta at vi kan vise at $ \mathbb{R} $ og $ \mathbb{N} $ er like store, og så vise at det fører til en motsigelse.

Vi kan begynne med noe som kanskje går litt mot intuisjonen. Det er nemlig like mange rasjonelle tall som naturlige tall. For å vise dette må vi vise at vi har en 1-1-korrespondanse mellom naturlige tall og rasjonale tall. Vi starter med å liste opp \say{alle} de naturlige og rasjonale tall. Vi setter rasjonale tall inn i en tabell:

\[ \mathbb{N} = 1, 2, 3 ... \quad\quad \mathbb{Q} = \begin{tabular}{c | c c c c}
  & 1 & 2 & 3 & ... \\
  \hline
1 & $ ^1/_1 $ & $ ^2/_1 $ & $ ^3/_1 $ & ... \\
2 & $ ^1/_2 $ & $ ^2/_2 $ & $ ^3/_2 $ & ... \\
3 & $ ^1/_3 $ & $ ^2/_3 $ & $ ^3/_3 $ & ... \\
\vdots & \vdots & \vdots & \vdots & $ \ddots $
\end{tabular} \]

Sånn rent intuitivt kan det se ut som at det er mange flere rasjonale tall enn naturlige tall, men hvis vi går på skrå i tabellen over $ \mathbb{Q} $ kan vi knytte hvert rasjonalt tall til ett naturlig tall, og hvert naturlige tall til ett rasjonalt tall:

\[ 1: \frac{1}{1} \quad\quad 2: \frac{2}{1} \quad\quad 3: \frac{1}{2} \quad\quad 4: \frac{3}{1} \quad ... \]

Vi har dermed vist at vi kan skape en 1-1-korrespondanse mellom $ \mathbb{N} $ og $ \mathbb{Q} $, og dermed er de like store. 

Vi skal nå se på $ \mathbb{N} $ og $ \mathbb{R} $, altså de reelle tallene. Vi ønsker å prøve å skape en lignende 1-1-kobling som vi gjorde for de rasjonale tallene. $ \mathbb{R} $ er ikke tellbart uendelig, derfor kan vi ikke liste opp alle tallene i $ \mathbb{R} $ på samme måte som før med en smart tabell (dette er forøvrig grunnen til at $ \mathbb{R} $ er større enn $ \mathbb{N} $, vi skal nå vise \emph{hvorfor}). En ting vi kan gjøre er å liste opp naturlige tall på en side, og tilfeldige og unike reelle tall på den andre. Det har seg faktisk slik at det er flere reelle tall mellom 0 og 1 enn det er naturlige tall tilsammen, det holder å liste opp tilfeldige reelle tall mellom 0 og 1. Vi får da en slik tabell:

\begin{center}
\begin{tabular}{c | c}
$ \mathbb{N} $ & $ \mathbb{R} $ \\
\hline
1 & 0.182947... \\
2 & 0.294817... \\
3 & 0.132849... \\
\vdots & \vdots
\end{tabular}
\end{center}

Vi kan tenke oss at hvis vi fortsetter slik i all evighet vil vi til slutt ende opp med en 1-1-korrespondanse mellom $ \mathbb{R} $ og $ \mathbb{N} $. Vi vil jo aldri gå tom for naturlige eller reelle tall å ta av. Det viser seg likevel å være feil, og det er her Cantors diagonaliseringsargument kommer inn:

Vi kan nemlig lage et nytt reellt tall som ikke eksisterer i denne tabellen. Hvis vi tar første siffer fra første tall og legger til 1, andre siffer fra andre tall og legger til 1, også videre. Generelt tar vi det $ i $-te sifferet og legger til 1. Hvis tallet er 9 kan vi gå til 0\footnote{Formelt kan vi bruke klokkeaddisjon (moduloaddisjon): $ (9 + 1) \mod{10}  = 0 $}. Gjør vi det for denne tabellen får vi:

\[ 0.203...  \]

Siden dette tallet er ulikt tall nummer $ i $ i tabellen på $ i $-te siffer vet vi at det ikke er inneholdt i tabellen, og vi har dermed vist at det ikke kan lages en 1-1 korrespondanse mellom $ \mathbb{R} $ og $ \mathbb{N} $. Følgelig er $ \mathbb{R} $ større enn $ \mathbb{N} \hfill\qed$ 


\newpage
\subsubsection{Antall noder i et binært tre}
Vi skal vise at antall noder $ n $ i et binært tre er begrenset slik:
\begin{equation*}
n \leq 2^{h} - 1
\end{equation*}
Vi skal vise dette ved induksjon.

Vi starter med å vise at formelen gjelder for et tre med høyde $ h=1 $:
\[ n = 2^1 - 1 = 1 \]
som åpenbart stemmer siden et tre med høyde 1 har kun 1 node, nemlig rota.

Vi antar at formelen gjelder for alle $ h $ opp til $ n $, og skal vise at det impliserer at den også gjelder for $ n+1 $. Vi har et tre $ T_1 $ med høyde $ n $:
\begin{center}
\begin{tikzpicture}[sibling distance=8em]
\node[isosceles triangle, draw=black, align=center, minimum height=0.5cm, minimum width=1cm, shape border rotate=90, anchor=north]{$ T_1 $};
\end{tikzpicture}
\end{center}
Vi antar at formelen stemmer, og dermed at antall noder i $ T_1 $ maksimalt er $ 2^n - 1 $. Vi skal nå øke høyden til $ n+1 $. Vi lager en ny rot, med to fulle subtrær som barn:
\begin{center}
\begin{tikzpicture}[sibling distance=8em]
\node[shape=circle, draw, align=center]{r} 
	child {node[isosceles triangle, draw=black, align=center, minimum height=0.5cm, minimum width=1cm, shape border rotate=90, anchor=north]{$ T_1 $}}
	child {node[isosceles triangle, draw=black, align=center, minimum height=0.5cm, minimum width=1cm, shape border rotate=90, anchor=north]{$ T_2 $}};
\end{tikzpicture}
\end{center}
Fra formelen har vi at $ T_1 $ og $ T_2 $ maksimalt har $ 2^n-1 $ noder hver. Legger vi sammen antall noder i treet nå får vi:
\[ n ~=~ \text{noder i } T_1 + \text{noder i } T_2 + 1 ~\leq~ 2(2^n -1) + 1 ~=~ 2^{n+1} - 1 \]

Oppsummering: Vi har vist at hvis formelen gjelder for $ h=n $ gjelder den også for $ h=n+1 $, vi har vist at den gjelder for $ h=1 $, og dermed har vi vist at den gjelder for alle $ h \in \mathbb{N} \hfill\qed$