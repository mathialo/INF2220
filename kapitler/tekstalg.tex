\section{\color{red}Tekstalgoritmer}
	Generelt problem: Vi er interesserte i å finne ut om en streng er en substreng av en annen streng. Vi kaller substringen for en nål $i$, av lengde $m$, og stringen for en høystakk $h$, av lengde $n$.

	\subsection{Brute force}
		Brute force-metoden er som navnet tilsier ikke spesielt gjennomtenkt: Til å begynne med sjekker vi hver karakter i høystakken mot den første i nåla. Hvis vi har likhet sjekker vi det neste tegnet i høystakken mot det neste i nåla. Vi fortsetter slik til vi enten finner hele nåla i høystakken eller til vi har ulikhet, og i så fall fortsetter vi med å sjekke det neste tegnet i høystakken mot det første i nåla.

	\subsubsection{Analyse}
		Worst case så får vi mismatch $(n-m)$ ganger og suksess $(n-m)+1$ ganger, Totale sammenligninger er $((n-m)+1\times m)$ som gir kjøretid $O(n^2)$.

	\subsection{\color{red}Boyer-Moore}
		Dette er en rask substringalgoritme. Med Boyer-Moore preprosserer vi nåla før vi begynner å søke. Vi regner ut hvor mange hopp vi kan gjøre for hvert enkelt tegn i nåla ved mismatch. Dette er bad character shift. Good character shift også... Algoritmen starter med å sjekke halen på nåla og ved eventuell match beveger seg mot venstre. Fortsetter så enten til hele nåla er matchet eller til mismatch, hvor vi i så fall beveger oss \verb|badShift[i]| bortover.

	\subsubsection{\color{red}Bad character shift}
		Vi beregner avstand til neste gang nål er på linje med høystakk\ref{huffman}.
	\subsubsection{\color{red}Good suffix shift}
		Vi bruker antall match før mismatch for å finne skip-avstand.

	\subsection{\color{red}Huffmankoding}\label{huffman}
		\begin{enumerate}
			\item Lager frekvenstabell for alle tegn som forekommer i teksten.
			\item Betrakt hvert tegn som en node og legg alle noder i en prioritetskø (heap). Se \ref{heap}.
			\item Mens P har mer enn ett element:
				%\begin{itemize}
					
				%\end{itemize}
		\end{enumerate}